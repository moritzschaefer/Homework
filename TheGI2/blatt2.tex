\documentclass[a4paper]{scrartcl}
% Source: http://www.howtotex.com/templates/two-column-journal-article-template/

% Source: http://www.howtotex.com/templates/two-column-journal-article-template/

% SETTINGS BEGIN

\usepackage{assignment}
\usepackage[hmarginratio=1:1,top=25mm,left=25mm]{geometry}
\usepackage{arydshln}


\usepackage{wasysym}

\usepackage{ stmaryrd }
\usepackage{tikz}
\usetikzlibrary{fit,shapes,arrows}

\newcommand{\impldef}[1]{\stackrel{\mathrm{#1}}\Rightarrow}
\newcommand{\biimpldef}[1]{\stackrel{\mathrm{#1}}\Leftrightarrow}
\newcommand{\ldot}{\;.\;}
\newcommand{\N}{\mathbb{N}}
\newcommand{\R}{\mathbb{R}}
\newcommand{\Ps}{\mathcal{P}}

\renewcommand{\labelitemi}{$-$}




\assignment{Hausaufgabe 2}{Theoretische Grundlagen der Informatik 2 SS 2013\\TU Berlin}
\members{Max Gotthardt(), Marco Morik(), Moritz Schäfer (350651)}{Tutor: }

\begin{document}
\maketitle

\subsection{Aufgabe 1}
\begin{description}
  \item[a)]
    Die Aussage ist falsch. Ein Beispiel:
    \begin{align}
      \Sigma = \{a\} A = \{a\} \\
      (\overline{A})^* &=  \Sigma^* \setminus \{a\} \\
      \overline{A^*} &= \Sigma^* \setminus A^* = \emptyset \\
    \end{align}
  \item[b)] 
    Die Aussage ist falsch. Beispiel:
    \begin{align}
      A = \{a\} B = \{b\} \\
      A^* \cup B^* &= \{a,aa,aaa,\dots,b,bb,bbb,\dots\} \\
      (A \cup B)^* &= \{a,b,ab,ba,aba,bab,\dots\} \\
    \end{align}
  \item[c)]
    Die Aussage ist falsch. Beispiel:
    \begin{align}
      A = \{a,b\} B = \{ab\} \\
      A \cap B &= \emptyset \\
      A^* \cap B^* &= \{ab, abab, ababab, \dots\} \\
    \end{align}
\end{description}

\end{document}

