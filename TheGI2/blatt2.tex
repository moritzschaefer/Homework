\documentclass[a4paper]{scrartcl}
% Source: http://www.howtotex.com/templates/two-column-journal-article-template/

% SETTINGS BEGIN

\usepackage{assignment} \usepackage[hmarginratio=1:1,top=25mm,left=25mm]{geometry}
\usepackage{arydshln}
\usepackage{qtree}


\usepackage{wasysym}

\usepackage{ stmaryrd }
\usepackage{tikz}
\usetikzlibrary{fit,shapes,arrows}

\newcommand{\impldef}[1]{\stackrel{\mathrm{#1}}\Rightarrow}
\newcommand{\biimpldef}[1]{\stackrel{\mathrm{#1}}\Leftrightarrow}
\newcommand{\ldot}{\;.\;}
\newcommand{\N}{\mathbb{N}}
\newcommand{\R}{\mathbb{R}}
\newcommand{\Ps}{\mathcal{P}}

\renewcommand{\labelitemi}{$-$}




\assignment{Hausaufgabe 2}{Theoretische Grundlagen der Informatik 2 SS 2013\\TU Berlin}
\members{Max Gotthardt(), Marco Morik(), Moritz Schäfer (350651)}{Tutor: }

\begin{document}
\maketitle

\subsection{Aufgabe 1}
\begin{description}
  \item[a)]
    Die Aussage ist falsch. Beispiel:
    \begin{align*}
      \Sigma = \{a\}; \  A = \{a\} \\
      (\overline{A})^* &=  \Sigma^* \setminus \{a\} \\
      \overline{A^*} &= \Sigma^* \setminus A^* = \emptyset \\
    \end{align*}
  \item[b)] 
    Die Aussage ist falsch. Beispiel:
    \begin{align*}
      A = \{a\} ; \  B = \{b\} \\
      A^* \cup B^* &= \{a,aa,aaa,\dots,b,bb,bbb,\dots\} \\
      (A \cup B)^* &= \{a,b,ab,ba,aba,bab,\dots\} \\
    \end{align*}
  \item[c)]
    Die Aussage ist falsch. Beispiel:
    \begin{align*}
      A = \{a,b\} ; \  B = \{ab\} \\
      A \cap B &= \emptyset \\
      A^* \cap B^* &= \{ab, abab, ababab, \dots\} \\
    \end{align*}
\end{description}
\subsection{Aufgabe 2} % (fold)
\label{ssub:Aufgabe 2}
\begin{description}
  \item[a)] 
    \begin{itemize}
      \item $ S \rightarrow aS \rightarrow aaS \rightarrow aaTb \rightarrow aabb $
      \item $ S \rightarrow aS \rightarrow aaS \rightarrow aaTb \rightarrow aaTbb \rightarrow aabb $
      \item $ S \rightarrow aS \rightarrow aTb \rightarrow aabb $
    \end{itemize}
  \item[b)] 
    \begin{itemize}
      \item \Tree [.S a [.S a [.S [.T b ] b ] ] ]
      \item \Tree [.S a [.S a [.S [.T  [.T  epsilon ] b ] b ] ] ]
      \item \Tree [.S a [.S [.T a b ] b ] ]
    \end{itemize}
\end{description}
% subsection Aufgabe 2 (end)
\subsection{Aufgabe 3} % (fold)
\label{sub:Aufgabe 3}
\begin{description}
  \item[1.] $ \emptyset^* = \{\epsilon\} \{\epsilon\}^* = \{\epsilon\} $
  \item[2.] $  A ^ * = \bigcup _ { n \geq 0 }  A^n $\\
    Nach 1. ist $\emptyset^*$ und $j\{\epsilon\}^*$ nicht unendlich. Enthält A hingegen ein Element, gibt es unendlich verscheidene Kombinationen dieses Elementes $n\geq0$ enthält $\infty$ viele $n$.
\end{description}

% subsection Aufgabe 3 (end)
\subsection{Aufgabe 4} % (fold)
\label{sub:Aufgabe 4}
$ G =  \{\{S,M,K\},\{a,b,c\}, P, S\} $ \\
\begin{tabbing}

$ P = \{ $\=$ S \rightarrow aKbbMc, $\\
         \=$ K \rightarrow aKb | \epsilon | Kb, $\\
         \=$ M \rightarrow bMc | \epsilon | bM \} $
\end{tabbing}
% subsection Aufgabe 4 (end)

\end{document}

